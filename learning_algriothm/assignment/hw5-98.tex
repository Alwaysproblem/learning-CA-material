\documentstyle[12pt]{article}
\input{psfig.sty}
\input epsf
\oddsidemargin 0cm      % this is margin in addition to the usual 1 inch
\evensidemargin 0cm
\textwidth 16.5cm

%\oddsidemargin -0.5cm      % this is margin in addition to the usual 1 inch
%\evensidemargin-0.5cm
%\topmargin -1.9cm
%\topmargin 0.0cm
%\baselineskip 0.0cm

%\textwidth 16.5cm
%\textwidth 17.0cm
%\textheight 24.7cm

\newcommand{\bi}{\begin{itemize}}
\newcommand{\ei}{\end{itemize}}
\newcommand{\be}{\begin{enumerate}}
\newcommand{\ee}{\end{enumerate}}
\newcommand{\tand}{\wedge}
\newcommand{\tor}{\vee}
\newcommand{\union}{\cup}
\newcommand{\intersection}{\cap}
\newcommand{\ch}{$\surd$}
\newcommand{\pgm}[1]{{\sc #1}}
\newcommand{\la}{\leftarrow}
\newcommand{\ra}{\rightarrow}

\begin{document}
\begin{center}
\large
{\bf Machine Learning 15-681/781}\\
Tom Mitchell, Fall 1998
\end{center}

\begin{center}
\large  Homework Assignment 5\\
Due Friday Dec 4, 1998 (turn in to Jean Harpley, Wean 5313)\\
\ \\
We have covered the material you need to answer questions 1,2, and 3, in
previous lectures.

\end{center}

\be
\item 
Radial Basis Function networks\footnote{Important: Please note the equation on
line three of page 239 is in error.  A minus sign should be added to the
exponent.  A full set of errata for the textbook is available at
www.cs.cmu.edu/$\sim$tom/mlbook.html}.  Consider a variant of the
distance-weighted $k$ nearest neighbor algorithm defined in the text.  In
particular, consider a variant which we will call Gaussian distance weighted
nearest neighbor (GDWNN), in which (1) the classification for a query instance
is calculated using equation 8.2 from the textbook, (2) $k$ is set equal to
the number of training examples, and (3) the distance weight of equation 8.3
is redefined as follows:
\[ w_i \equiv e^{-\frac{1}{2}d^2(x_q,x_i)} \]

True or false? For every possible set of training examples $D$, it is possible
to construct a Radial Basis Function network that is equivalent to the GDWNN
classifier for $D$.  (here ``equivalent to'' means that the two classifiers
agree on the classification of every possible query instance).  If true,
describe precisely your procedure for constructing the RBF network.  If false,
explain why this is impossible.  (to convert your RBF network output to a
boolean value, interpret outputs greater than or equal to zero as True, and
outputs less than zero as False.
\item Exercise 8.3 from the textbook.
\item Exercise 9.4
\item Exercise 13.2
\ee

\end{document}
